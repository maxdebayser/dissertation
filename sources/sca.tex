\section{The Model}
In chapter \ref{chap:components} we have described how effective components based on services can be, as they need to be deployed only
once and can be used by many loosely-coupled clients. But the usual service oriented middleware approach has several disadvantages.
The problem is that large systems tend to grow organically over time. Several small systems are first built to handle specific needs
but over time these have to be integrated into the larger system. However at this point difficulties tend to arise because
those services may have been built using different and incompatible technologies. The integration is difficult due to two
factors. The first one is that SOA technologies are not directly interoperable. The second factor is that many SOA technologies
force a programming style that unnecessarily tangles integration code with business logic, which prevents one from porting
existing services to a new implementation technology. Pichler \emph{et al.} identified several problems with the \texttt{EJB}
component model, but it applies to others like \texttt{CCM} as well \cite{Pichler}. They identified the following problems:
\textbf{Lack of Tailorability.}. The \texttt{EJB} specification does not define a way to extend container with new services or configure
existing ones. This forces component developers to address crosscutting concerns in the component implementation leading to application
code that is unnecessarily tangled with infrastructure code. In addition, it is not possible to remove unneeded service from
the \texttt{EJB} container, forcing the deployment of the entire \texttt{EJB} environment. \textbf{Lack of checking and enforcement.} The \texttt{EJB}
specification expects the programmer to follow several programming rules and idioms that cannot be enforced by the compiler.
In addition, common use of \texttt{EJB}'s \texttt{API} involves loss of static type safety. \textbf{Insufficiency.} It is not possible to host
ordinary \texttt{Java} classes in an \texttt{EJB} container. To be supported by the container, classes must be developed especially for EJB.

The Service Component Architecture (\texttt{SCA}) standard is an attempt to improve this situation. It specifies a framework
where many different communication and component implementation technologies can be integrated. It achieves this
with a modular design that decouples components from the underlying infrastructure. At the same time it allows
interoperability with legacy services also avoiding unnecessary coupling to any communication protocol. New
technologies and languages are supported with extensions to the core runtime.

In \texttt{SCA}, components are loosely coupled to the infrastructure because it abstracts away the communication protocol
and enforces a declarative handling of component dependencies. A component, when initialized, never searches
actively for other services it depends on. Instead, references to the required services are injected by the framework
during the initialization of a component. In other words, the core runtime uses a dependency injection model to
configure and connect components. However, how the dependency injection manifests itself at the component implementation
level depends on the implementation language and on the design decisions taken for that particular language extension.
As discussed below, the \texttt{Java} language binding supports dependency injection at the implementation level and therefore
properties and references are represented as class fields. The \texttt{C++} language binding however, does not and the components
are forced to look up property values and references using the binding's \texttt{API}.

In \texttt{SCA}, composition is directed by a declarative configuration file. This configuration file, also called \emph{composite file},
is written in an \texttt{XML} language specified by the \texttt{SCA} standard. This language is meant to be extensible to give the necessary
freedom of adding new elements and attributes to the extension developer. This is possible because \texttt{SCA} validates the
composite files using a composition of \texttt{XSD} schemas.

\subsection{Components}

The most important elements of \texttt{SCA} are, of course, components. They provide services and can have configuration properties
and dependencies on other services. Components can be implemented in any language if there is an extension for it. A component
implementation extension is basically a plug-in that is responsible for loading a component in a language-specific way, applying
the configuration and intercepting requests from and to the component and expressing them in a language-specific way like method
calls, for example. Listing \ref{lst:calcservice} shows a simple example of a component written in \texttt{Java}. The component is an instance
of the \texttt{CalculatorServiceImpl} class. It provides a single service that follows the contract represented by the \texttt{CalculatorService}
interface. It has one configuration property, \texttt{coefficient}, that is configured by the run time using setter injection.
This component also depends on an external service that follows the contract represented by the \texttt{DivideService} interface.
The service is represented as a \texttt{Java} object that implements this interface and its reference is also provided to the component
by the way of setter injection.

\begin{listing}
\begin{minted}[linenos,fontsize=\footnotesize]{java}
@Remotable
public interface CalculatorService {
  public double divide(double n1, double n2);
}

public class CalculatorServiceImpl implements CalculatorService {

  private DivideService divideService;
  
  private double coefficient;
  
  @Property
  public void setCoefficient(double c) {
    this.coefficient = c;
  }

  @Reference
  public void setDivideService(DivideService dS) {
     this.divideService = dS;
  }

  public double divide(double n1, double n2) {
     return divideService.divide(n1, n2);
  }
}
\end{minted}
\caption{A simple component}
\label{lst:calcservice}
\end{listing}
\begin{listing}
\begin{minted}[linenos,fontsize=\footnotesize]{xml}
<composite xmlns="http://www.osoa.org/xmlns/sca/1.0" name="CalculatorComposite">

   <component name="DivideComponent">
      <implementation.java class="org.example.DivideServiceImpl" />
   </component>

   <component name="CalculatorComponent">
      <implementation.java class="org.example.CalculatorServiceImpl" />
      <property name="coefficient">3.14</property>
      <reference name="divideService" target="DivideComponent" />
   </component>

</composite>
\end{minted}
\caption{A sample configuration file}
\label{lst:calcservicexml}
\end{listing}

Listing \ref{lst:calcservicexml} shows a simple configuration file for the calculator component of listing \ref{lst:calcservice}.
We use the \texttt{XML} element \texttt{component} to instantiate a component. In this example we instantiate two components:
\texttt{DivideComponent} and \texttt{CalculatorComponent}

Components in \texttt{SCA} are stateless by default and can be instantiated and destroyed on demand by the runtime. Furthermore, the
\texttt{SCA} runtime guarantees that no instance will receive concurrent method calls. If there is more than one incoming call, the
runtime creates a separate instance for each one. It is also possible to create components that maintain state and persist
for the lifetime of the parent composite. In this case it is up to the component developer to make sure that it is thread-safe.

Composition in \texttt{SCA} is recursive, one can use a composite as a single component. The \texttt{SCA} runtime provides a special implementation
type that loads a composite \texttt{XML} files, performs all the connections and treats the results a component. As shown in listing
\ref{lst:composite}, at line 9, all there is to do is to use the \texttt{$<$implementation.composite$>$} element to instruct the runtime
to load a composite.

\begin{listing}
\begin{minted}[linenos,fontsize=\footnotesize]{xml}
<composite xmlns="http://www.osoa.org/xmlns/sca/1.0" name="StoreComposite">

   <component name="StoreComponent">
      <implementation.java class="org.example.DivideServiceImpl" />
      <reference name="calculatorService" target="CalculatorComponent" />
   </component>

   <component name="CalculatorComponent">
      <implementation.composite name="CalculatorComposite" />
   </component>

</composite>
\end{minted}
\caption{Composite implementation type}
\label{lst:composite}
\end{listing}

As services and references of a composite component we can use selected services and references of internal components
that were left unconnected. To instruct the runtime to expose a service for composition outside of the containing composite
we \emph{promote} as shown in listing \ref{lst:calcservicexmlpromote}, line 3.

\begin{listing}
\begin{minted}[linenos,fontsize=\footnotesize]{xml}
<composite xmlns="http://www.osoa.org/xmlns/sca/1.0" name="CalculatorComposite">
    ...
    <service name="CalculatorService" promote="CalculatorComponent/CalculatorService">
</composite>
\end{minted}
\caption{Service Promotion}
\label{lst:calcservicexmlpromote}
\end{listing}

A connection between components is called a \emph{wire} in \texttt{SCA}. A single reference can be wired to several services
but it can get unwieldy to list several connections inside the \texttt{$<$reference$>$} element. To improve the readability
of \texttt{XML} configuration files, one can use the \texttt{$<$wire$>$} element to connect components, as shown in listing 
\ref{lst:calcservicexmlwires}

\begin{listing}
\begin{minted}[linenos,fontsize=\footnotesize]{xml}
<composite xmlns="http://www.osoa.org/xmlns/sca/1.0" name="CalculatorComposite">

   <component name="DivideComponent">
      <implementation.java class="org.example.DivideServiceImpl" />
   </component>

   <component name="CalculatorComponent">
      <implementation.java class="org.example.CalculatorServiceImpl" />
      <property name="coefficient">3.14</property>
   </component>
   
   <wire source="CalculaterService/divideService" target="DivideComponent" />

</composite>
\end{minted}
\caption{A sample configuration file}
\label{lst:calcservicexmlwires}
\end{listing}

\subsection{Bindings}

In many cases an enterprise system will depend on existing external services that are not running on a \texttt{SCA} infrastructure.
Conversely, it might be necessary to expose a component's services to the outside world, without requiring external clients
to run on \texttt{SCA}. Instead of leaving component developers on their own to solve this issue, \texttt{SCA} specifies
a transparent way of connecting components to external entities. From the component's developer point of view, it makes
no difference if it is connected to an external service or to another component. These external reference or service
connections are called bindings.

When a binding is declared, details such as the address and the communication protocol must be known. For example, if we
wanted to expose the \texttt{CalculaterService} as a web service we could do as shown in listing \ref{lst:calcservicexmlbinding1}.

\begin{listing}
\begin{minted}[linenos,fontsize=\footnotesize]{xml}
<composite xmlns="http://www.osoa.org/xmlns/sca/1.0" name="CalculatorComposite">
    ...
    <service name="CalculatorService" promote="CalculatorComponent/CalculatorService">
      <binding.ws uri="http://math.com/services/calculator" />
    </service>
</composite>
\end{minted}
\caption{Service bindings}
\label{lst:calcservicexmlbinding1}
\end{listing}

Services can be made available through several bindings at the same time. All it takes is adding more binding configurations
inside the \texttt{$<$service$>$} element.

The configuration for reference bindings is very similar to the one for services. Listing \ref{lst:calcservicexmlbinding2}
continues with the \texttt{CalculaterService} example, but this time the calculator component uses an external service
instead of the local \texttt{DivideComponent}.

\begin{listing}
\begin{minted}[linenos,fontsize=\footnotesize]{xml}
<composite xmlns="http://www.osoa.org/xmlns/sca/1.0" name="CalculatorComposite">

   <component name="CalculatorComponent">
      <implementation.java class="org.example.CalculatorServiceImpl" />
      <property name="coefficient">3.14</property>
   </component>
   
   <reference name="divideService" promote="CalculatorComponent/divideService">
      <binding.ws uri="http://math.com/services/divide" />
   </reference>
   
</composite>
\end{minted}
\caption{Reference bindings}
\label{lst:calcservicexmlbinding2}
\end{listing}

As with component implementations, \texttt{SCA} can support any communication protocol as long as there is a plug-in for it.

In reality, even internal connections always go through a binding. In the absence of an explicit instruction, components
are connected using \texttt{SCA} the default binding. This can be made explicit using the \texttt{$<$binding.sca$>$} element.
The default binding can be overridden in the configuration file. For example, we can instruct the \texttt{SCA} runtime to
connect to components using \texttt{Java RMI}, as shown in \ref{lst:calcservicexmlbinding3}

\begin{listing}
\begin{minted}[linenos,fontsize=\footnotesize]{xml}
<composite xmlns="http://www.osoa.org/xmlns/sca/1.0" name="CalculatorComposite">

   <component name="DivideComponent">
      <implementation.java class="org.example.DivideServiceImpl" />
   </component>

   <component name="CalculatorComponent">
      <implementation.java class="org.example.CalculatorServiceImpl" />
      <property name="coefficient">3.14</property>
      <reference name="divideService" target="DivideComponent">
	<binding.rmi />
      <reference>
   </component>

</composite>
\end{minted}
\caption{A sample configuration file}
\label{lst:calcservicexmlbinding3}
\end{listing}

It is not generally recommended to override the default binding in connections between components because it restricts
the runtime's freedom to choose the most appropriate binding. For instance, the runtime could choose to use direct method
calls for components in the same address space.

\subsection{Interfaces}
%TODO, falar de interfaces extensions
In \texttt{SCA} service contracts can be seen as object-oriented interfaces: a named set of methods. In most object-oriented programming
languages and middleware platforms interfaces are either directly supported, as in \texttt{Java} or \texttt{CORBA}, or simulated using well-known
conventions, as in \texttt{C++}. Although this concept is natural in object-oriented programming, it does not necessarily map in the same
way on every language and thus it can be challenging to make components written in different languages interoperable.

Older object-oriented middleware platforms tried to address this problem by requiring interfaces to be written in a
interface description language (IDL). These interfaces would then be processes by a tool to generate abstract base classes
which implementation classes would inherit from. The problem with this approach is that it encourages a strong coupling
of components to that particular middleware, thereby reducing its portability.

As \texttt{SCA} tries to avoid platform lock-in, it has taken an entirely different approach. Instead of requiring the use
of a implementation language-independent IDL components can use interfaces written in the implementation language.
The only requirement is that the interface on the client side must be a subset of the one at the server side and that
argument types can be mapped cleanly from one language to the other. The difference between the two approaches is
similar to the difference between static typing and structural typing in programming languages.

A target service interface is considered compatible with the reference interface if it defines the same set, or a superset,
of operations. The operation names must be the same as well as the parameter types, the parameter ordering and the return
type.
%TODO elaborar
In some cases when creating a service binding, the \texttt{SCA} runtime can create a WSDL interface description from the \texttt{Java} interface.


\section{SCA and dependency injection}

\texttt{SCA} was proposed at a time when the \texttt{Java} enterprise developer community had already experienced the complexity of component
platforms such as \texttt{EJB} and moved on to simpler lightweight dependency injection containers like Spring. For this reason
\texttt{SCA} tries to follow the same principles to avoid issues such as container coupling, lack of portability and interoperability.
In a sense, everything up to the communication stack is injected rather than hard-coded.

The core runtime does its part to allow component to be configured using dependency injection. At start-up it reads the
composite assembly file, locates the required implementation, interface and binding extensions and configures them
according to the user's instructions. From this point on, it is up to the implementation extension to provide an environment
suitable for the development of components that are configured externally.

OSOA also has standards for implementation extensions for \texttt{Java} and \texttt{C++} \cite{SCA}. While \texttt{Java}'s binding fully supports
dependency injection, \texttt{C++}'s standard does not. The reason, as explained in chapter \ref{chap:ioc}, is that runtime
introspection is necessary to implement a generic container that can handle objects of classes unknown at compile-time.
While \texttt{Java} has a built-in introspection support, \texttt{C++} does not. For this reason, the \texttt{SCA} \texttt{C++} standard requires components
to use an \texttt{SCA}-specific \texttt{API} to retrieve configuration values and service references as needed. This leads to an unfortunate
situation where \texttt{C++} components are almost independent on the underlying infrastructure, but not enough to be reused in
other contexts.

The dependency on a \texttt{API} also implies a dependency at the module level between the components module and the \texttt{API}'s module.
So even if the \texttt{API} is only an abstract facade that could allow several implementations, the component's module can not be deployed
without the \texttt{API}'s module.

Actually there is an \texttt{SCA} \texttt{C++} container, Trentino \cite{Trentino}, that supports a limited form of inversion of control as it is
built on top of \texttt{PocoCapsule} \cite{PocoCapsule}. But as discussed in chapter \ref{chap:ioc}, \texttt{PocoCapsule} uses a configuration
file, in this case the composite file, as input to the injection code generator and consequently this adapter code must be
recompiled every time there is a significant change in the configuration file. Nonetheless, this scheme allows to make minor
changes to configurations such as changing a configuration value. An additional shortcoming of this scheme is that
it is impossible to introspect interfaces at runtime to generate representations in another language such as \texttt{WSDL} and \texttt{CORBA IDL}.

\section{Tuscany native}

\texttt{Apache Tuscany} is a project hosted by the Apache Foundation \cite{Tuscany}, \cite{Laws}. It includes one implementation written
in \texttt{Java} that supports components written in \texttt{Java}, \texttt{BPEL}, \texttt{Python} and many messaging protocols such as \texttt{RMI}, \texttt{CORBA}, \texttt{SOAP} and \texttt{JMS}.
This project also includes a more limited \texttt{SCA} runtime written entirely in \texttt{C++}, which includes a \texttt{C++} implementation extension.
However this extension does not support dependency injection. As described in the previous section, components have to use
\texttt{SCA}'s \texttt{API} to retrieve the configuration properties and service references.

\texttt{Apache Tuscany} has a modular architecture, reflecting \texttt{SCA}'s extensible model. Because \texttt{SCA} is designed to support many
different implementation languages and messaging protocols, it is designed as a small runtime core with plug-in extensions.

\texttt{Tuscany} has a registry for each kind of extension. During the runtime's initialization, it searches the filesystem for
extensions. Basically an extension is deployed in a fixed directory structure at a given path and must contain a shared
library file containing the extension's implementation and a \texttt{XSD} Schema file to verify extension-specific syntax.
When an extension is loaded, an entry point function of the shared library is called to register the extension in
the appropriate registry.

Because different extensions might require different configurations, \texttt{SCA}'s \texttt{XML} assembly language is designed to be extensible.
For example, the \texttt{$<$implementation.java$>$} element has a different syntax than the \texttt{$<$implementation.cpp$>$}
due to differences between the two languages. In \texttt{Tuscany}, composite files are verified using an \texttt{XSD} schema. This schema is
composed of a main file for the core syntax and each extension provides an additional file that determines it's specific syntax.
%TODO encaixar esse negocio em algum lugar e usar a figura das extensoes plugaveis do tuscany
There can be extensions for implementations, interfaces, data bindings, messaging protocols and policies.



%TODO Build the declarative model.
% -Components
% -Bindings + CompositeReferences + Wires

When the composite file is read, the core runtime builds a graph of classes that roughly corresponds to the
declarative structure of the \texttt{XML} file. This graph is then handed to the implementation extension which uses it
to load and configure the components accordingly.

The interaction between the core runtime and an extension happens through a set of abstract base classes. For example, the
implementation registry consists of pointers to objects that implement the \texttt{ImplementationExtension} base class.
\texttt{ImplementationExtensions} are builder objects that construct objects that implement the \texttt{ComponentType}
interface.

\texttt{ComponentType} objects are responsible for taking the declarative model of a composition and returning a configured
instance. This configured instance is composed of a collection of endpoints that implement interfaces such as
\texttt{ServiceBinding}, \texttt{ReferenceBinding}, \texttt{ServiceProxy} and \texttt{ServiceWrapper}. The
runtime then uses these endpoints to compose components and relay request messages between them.

The \texttt{ServiceWrapper} is an interface for objects that receive serialized parameter packs and relay them to a service.
These serialized messages are instances of the \texttt{Operation} class. Both implementation extension and binding extensions
implement this interface. In the case of an implementation extension, the parameters are de-serialized and the component's
method is invoked. In the case of a binding extension the parameters a converted to a wire format and sent to a remote component. 

The \texttt{ServiceProxy} is an interface for objects that receive a method invocation, serializing it and
calling a service wrapper. In the case of implementation extension, the source of the invocation can be a direct method
call by a component's implementation. In the case of a binding extension, the proxy object might listen for requests coming
in from a network interface.

A \texttt{ServiceBinding} is represents a service endpoint of a component. It has a reference to a \texttt{ServiceWrapper}
object that is used to effectively invoke a component's method.

A \texttt{ReferenceBinding} represents a reference endpoint of a component. It has a method to make the connection
between components that receives as parameter a \texttt{ServiceBinding}. This method must then create a service proxy
to invoke the \texttt{ServiceWrapper} that is held by the \texttt{ServiceBinding}.

\section{Proposed changes and implementation}

As previously discussed, \texttt{Tuscany}'s \texttt{C++} implementation extension does not support dependency injection. Our proposal is to
create another \texttt{C++} extension that uses the \texttt{SelfPortrait} library presented in chapter \ref{chap:reflection} in order
to support dependency injection. 

Naturally our extension is a modified version of the existing one. We will first describe how the original extension works
and then proceed to the changes that were necessary.
 
Most of the classes that are part of an implementation extension have a structural purpose so we will not discuss them in
detail. The real functionality of handling requests and responses is implemented in the \texttt{ServiceWrapper} and 
\texttt{ServiceProxy} classes. 

In \texttt{Tuscany}'s \texttt{C++} extension service wrappers are a three-level class hierachy. At the most abstract level is the
\texttt{ServiceWrapper} interface that contains the declaration of the \texttt{invoke} method. At the intermediate level of
this class hierarchy is the \texttt{CPPServiceWrapper} class that contains code that is common to all wrappers.
At the most concrete level is a class that is generated by a tool that the component developer must run before
compiling the component. This tool takes as input the name of the implementation class and the header file containing
the abstract base class representing the service interface. The class that is generated by this tool contains code
to invoke the methods of the implementation class and to create and destroy new instances. The method invocation
happens in the implementation of the abstract \texttt{invoke} method, that does the invocations based on the method names.
The generated source file also contains a global function whose name is based on the name of the component. When
the shared library containing the component implementation is loaded by the \texttt{C++} extension, this function is called
to create new instances of the service wrapper class.

Service proxies are also based on code generation. A tool is used to generate a class that implements a service interface
and provides an implementation for each method that serializes the arguments to an \texttt{Operation} object to invoke a service
wrapper. The same scheme of a global function with a predefined name is used to instantiate the proxies.

This scheme is very simple and effective but it has the shortcoming that no dependency injection is possible and
therefore components must use a special \texttt{API} provided by this implementation to retrieve configuration values and
service references. In addition the \texttt{API} and the generated proxy and wrapper classes use unsafe type conversions
possibly leading to invalid memory accesses.

To extend this same basic scheme to support dependency injection would require parsing the implementation classes
as well and reifying this type information. Basically the result would be close to a introspection support
and there would be no reason not to generalize it to a general-purpose introspection framework.

Our extension is based on a \texttt{SelfPortrait}, a general-purpose \texttt{C++} introspection library that is capable
of instantiating object, reading and writing attributes and calling methods. It also support dynamic proxies
similar to those supported by \texttt{Java}.

The structural classes in our extension are very similar to the ones of the existing extension. It is in the
implementation of proxies and wrappers that our extension diverges.

Our implementation of the \texttt{ServiceWrapper} class, \texttt{IoCServiceWrapper}, loads the shared library
that contains a component and tries to locate the meta-object that describes the implementation class using
the reflection \texttt{API}. This meta-object is then used to create and instance of that class using the default constructor.
We could have implemented constructor injection but that would require changing the \texttt{XML} language supported by
\texttt{Tuscany native}. The next step is to inject configuration values from the composite file based in the property names.
Attribute and setter injection are supported. The final step of the initialization phase is to inject the service
references. These can also be injected directly into public attributes or setter methods. The injection is done
locating the corresponding attribute or method meta-objects based on the property names. When the service wrapper
receives a request, it tries to find a meta-object for a method that has the same name as the operation and that
as an appropriate signature. It then invokes this method using the meta object.

Our implementation of the \texttt{ServiceProxy} class, \texttt{IoCServiceProxy} relies on dynamic proxies to provide
implementations of service interfaces. During the initialization phase, this class locates the class meta-object that
describes the interface class and creates a proxy for it. For each method, this proxy is configured to serialize its
arguments to an \texttt{Operation} object that is then handed to a service wrapper.

In addition to dependency injection, our scheme has the advantage that pre-compiled shared libraries can be loaded without
modifications. The introspection meta-data can be compiled to a second shared library and loaded separately.
The possibility of separating component shared libraries and reflection shared libraries is a trait that we share with
\texttt{Trentino}.

\section{Results}

To demonstrate the difference in component development using the existing \texttt{C++} extension and our
new extension we will use a sample component present in \texttT{Tuscany native}'s distribution. Listing
\ref{lst:calccpp} shows a calculator service that depends on an external service to perform
divisions. In this example, we can see that the component has to include an external \texttt{SCA} header
file. This include is necessary to get access to a \texttt{ComponentContext} object that is then
used to retrieve a reference to the division service. The \texttt{ComponentContext} context acts
as a key-value collection where the keys a the property names given in the composite file.
It is worthwhile to note that \texttt{ComponentContext::getService} method returns a \texttt{void*}
pointer that must be cast to the expected interface type. This cast can fail for two reasons, both
with catastrophic consequences. The first and most obvious reason for the conversion from \texttt{void*} to
\texttt{Divide*} to fail is that the pointer may actually point to something else.
This cast can also fail in a more subtle way if the \texttt{API} implementation is not careful. If the
\texttt{API} implementation puts a pointer to the implementation class in the key-value storage instead of
a pointer to the interface the conversion can fail. If the implementation class happens to
implement several interfaces, pointers to the same object but with different types can actually
point to memory locations that are a few words apart. This problem can be avoided if the runtime
uses a \texttt{static\_cast} to convert the implementation class pointer to the interface pointer
prior to its insertion in the \texttt{ComponentContext}.

\begin{listing}
\begin{minted}[linenos,fontsize=\footnotesize]{c++}
#include "Divide.h"
#include "Calculator.h"
#include "osoa/sca/ComponentContext.h"

class CalculatorImpl : public Calculator
{
public:
  CalculatorImpl() {}
  virtual ~CalculatorImpl() {}

  virtual float add(float arg1, float arg2) { return arg1 + arg2; }
  virtual float sub(float arg1, float arg2) { return arg1 - arg2; }
  virtual float mul(float arg1, float arg2) { return arg1 * arg2; }
  virtual float div(float arg1, float arg2) {
    float result = 0;

    osoa::sca::ComponentContext myContext = 
      osoa::sca::ComponentContext::getCurrent();

    Divide* divideService = reinterpret_cast<Divide*>(myContext.getService("divideService"));

    return divideService->divide(arg1, arg2);
  }
};
\end{minted}
\caption{A tuscany native component}
\label{lst:calccpp}
\end{listing}

Listing \ref{lst:calccppcomposite} shows the composite file for this component. Note that
the original \texttt{C++} implementation extension is selected using the \texttt{$<$implementation.cpp$>$}
element. Also notice that the name given in the \texttt{$<$reference$>$} element is used by
the runtime as a key in the \texttt{ComponentContext}

\begin{listing}
\begin{minted}[linenos,fontsize=\footnotesize]{xml}
<composite xmlns="http://www.osoa.org/xmlns/sca/1.0" name="sample.calculator">
  <component name="CalculatorComponent">
    <implementation.cpp library="Calculator" header="CalculatorImpl.h"/>
    <reference name="divideService">DivideComponent/DivideService</reference>
  </component>
        
  <component name="DivideComponent">
    <implementation.cpp library="Calculator" header="DivideImpl.h"/>
  </component>
</composite>
\end{minted}
\caption{The old composite file}
\label{lst:calccppcomposite}
\end{listing}

\texttt{Tuscany native} also requires a \texttt{componentType} file for each component instantiated in the composite file.
This file contains a description of a components services and references and their interface. Listing
\ref{lst:calccppccomponenttype} shows the \texttt{componentType} file for the calculator component. Notice
the \texttt{$<$interface.cpp$>$} element selecting the original \texttt{C++} interface extension.

\begin{listing}
\begin{minted}[linenos,fontsize=\footnotesize]{xml}
<componentType xmlns="http://www.osoa.org/xmlns/sca/1.0">

  <service name="CalculatorService">
    <interface.cpp header="Calculator.h"/>
  </service>

  <reference name="divideService">
    <interface.cpp header="Divide.h"/>
  </reference>
</componentType>
\end{minted}
\caption{The old component type file}
\label{lst:calccppccomponenttype}
\end{listing}

Having seen the original component sample, let us now turn our attention to the component
developed for our \texttt{C++} implementation extension. Listing \ref{lst:calcioccpp} shows the same
component with a few modifications. The only header file inclusions left are for application-
specific header files. We have added a pointer attribute to a \texttt{Divide*} object and a
setter method. The division method now simply uses this pointer to call the division service.
When this component is created the implementation extension takes the reference name and
tries to find an attribute with the same name or a setter method whose name follows \texttt{Java}'s
setter names rule. In this example it will use the \texttt{setDivideService} method for
setter injection. We have also added a public attribute for a property just to demonstrate
attribute injection. An important difference is that the runtime is now responsible for
checking that properties and reference are of the correct type. In our implementation this
is done at runtime. If the types do not match, an exception is thrown and the configuration
phase is aborted.

\begin{listing}
\begin{minted}[linenos,fontsize=\footnotesize]{c++}
#include "Divide.h"
#include "Calculator.h"

class CalculatorImpl : public Calculator
{
public:
  CalculatorImpl() {}
  virtual ~CalculatorImpl() {}

  virtual float add(float arg1, float arg2) { return arg1 + arg2; }
  virtual float sub(float arg1, float arg2) { return arg1 - arg2; }
  virtual float mul(float arg1, float arg2) { return arg1 * arg2; }
  virtual float div(float arg1, float arg2) { return divideService->divide(arg1, arg2); }
  virtual float circleArea(float radius)    { return pi*(radius*radius); }

  // setter injection
  void setDivideService(Divide* d) { divideService = d; }

  // attribute injection
  float pi = 3.14;
private:
  Divide* divideService;
};
\end{minted}
\caption{A tuscany native component with dependency injection}
\label{lst:calcioccpp}
\end{listing}

Listing \ref{lst:calcioccomposite} shows the modified composite file. The main difference
is the use of the \texttt{$<$implementation.ioc$>$} element to select our extension. The attributes
of this elements are the same except for the addition of the \texttt{class} attribute that must
contain the name of the implementation class that will be used to locate the class using the
\texttt{SelfPortrait} reflection \texttt{API}.

\begin{listing}
\begin{minted}[linenos,fontsize=\footnotesize]{xml}
<composite xmlns="http://www.osoa.org/xmlns/sca/1.0" name="sample.calculator">

  <component name="CalculatorComponent">
    <implementation.ioc library="Calculator" metadata="Calculator-md" header="CalculatorImpl.h" class="CalculatorImpl"/>
    <reference name="divideService">DivideComponent/DivideService</reference>
    <property name="pi">5</property>
  </component>
        
  <component name="DivideComponent">
    <implementation.ioc library="Calculator" metadata="Calculator-md" header="DivideImpl.h" class="DivideImpl"/>
  </component>

</composite>
\end{minted}
\caption{The new composite file}
\label{lst:calcioccomposite}
\end{listing}

The component type file also has suffered a few modifications, as shown in listing \ref{lst:calcioccomponenttype}.
The primary difference is the use of \texttt{$<$interface.ioc$>$} element to select our interface extension.
The \texttt{$<$class$>$} attribute contains the name of the interface class that is used to locate the
meta-object that represents it. This meta-object is used to build the dynamic proxies that are injected into
component references. Also the optional \texttt{metadata} attribute is used to load a separate shared library
that contains the compiled meta-data.

\begin{listing}
\begin{minted}[linenos,fontsize=\footnotesize]{xml}
<componentType xmlns="http://www.osoa.org/xmlns/sca/1.0" xmlns:xs="http://www.w3.org/2001/XMLSchema">

  <service name="CalculatorService">
    <interface.ioc header="Calculator.h" class="Calculator"/>
  </service>

  <reference name="divideService">
    <interface.ioc header="Divide.h" class="Divide"/>
  </reference>
  
  <property name="pi" type="xs:integer">3</property>
</componentType>
\end{minted}
\caption{The new component type file}
\label{lst:calcioccomponenttype}
\end{listing}

The difference at the physical level can be seen by the output of a command like \texttt{Linux's ldd}, that lists
dependencies on shared objects as shown in listing \ref{lst:deps}. From this listing we can see that the original
component depends on several of \texttt{Tuscany's} libraries while the new component depends solely on a few system
libraries. The meta-data shared library depends only on our introspection library and on the component file. It does
not depend on the \texttt{Tuscany} runtime so it can be reused in other contexts as well.


\begin{listing}
\begin{minted}[linenos,fontsize=\footnotesize]{text}
# Original
#> ldd libCalculator.so.0.0.0 
linux-vdso.so.1
libtuscany_sca.so.0
libtuscany_sca_cpp.so.0
libstdc++.so.6
libm.so.6
libc.so.6
libgcc_s.so.1
libtuscany_sdo.so.0
libpthread.so.0
/usr/lib/ld-linux-x86-64.so.2
libxml2.so.2
libdl.so.2
libz.so.1
liblzma.so.5

# New component
#> ldd libCalculator.so 
linux-vdso.so.1
libstdc++.so.6 => /usr/lib/libstdc++.so.6
libm.so.6 => /usr/lib/libm.so.6
libgcc_s.so.1 => /usr/lib/libgcc_s.so.1
libc.so.6 => /usr/lib/libc.so.6
/usr/lib/ld-linux-x86-64.so.2

# New component meta-data
#> ldd libCalculator-md.so 
linux-vdso.so.1
libselfportrait.so
libCalculator.so
libstdc++.so.6
libm.so.6
libgcc_s.so.1
libc.so.6
/usr/lib/ld-linux-x86-64.so.2
 \end{minted}
\caption{Physical dependencies}
\label{lst:deps}
\end{listing}

From a quantitative standpoint there a two things we can compare, the code size and the performance overhead.
In table \ref{tab:table1} we can see the sizes of the original component shared library file and the new one,
along with the meta-data file. While the component file without meta-data is smaller than the original file,
the meta-data file is quite large. There are several reasons why the meta-data files is bigger than the component
file. First of all, the component contains almost no code, so there is actually much more code to handle the
generic use of the component's interface than in actual methods. In a real-world application the component code
would be much bigger, leading to a smaller overhead. Another reason for this size is that the reflection library
uses template container classes from the standard library leading to the generation of code in the meta-data shared
object file. And finally, for each type of method argument an internal template of the \texttt{VariantValue} class
are instantiated leading to more code generation. However, as more interfaces and methods are introspected, it is
likely that the same type will be used many times, but generating code only once.

\begin{table}[ht]
\centering
\caption{Space comparison}
\label{tab:table1}
\begin{tabular}{ll}
\hline\noalign{\smallskip}
File & size\\
\noalign{\smallskip}
\hline
\noalign{\smallskip}
Original libCalculator.so & 29752 B\\
New libCalculator.so & 8000 B\\
Meta-data libCalculator-md.so & 138792 B\\
\hline
\end{tabular}
\end{table}

To compare the performance of the new and the old \texttt{Tuscany} binding we ran a test where the
division method was called 10 million times for each component version. We measured the processor use
using the \texttt{clock} \texttt{Unix} system call. In our extension we implemented an optimization where
direct method calls are used if the two components are written in \texttt{C++} and are hosted by the same
process. In table \ref{tab:table2} we can see that the direct method call is by far the fastest. In second
place comes \texttt{Tuscany's} original method call mechanism. The slowest method call is using dynamic proxies.
This happens because with the dynamic proxy approach there is much more work involved. \texttt{Tuscany's}
pre-compiled code just takes its arguments and constructs an \texttt{Operation} object and calls the connected
\texttt{ServiceWrapper}. The proxy version must find out dynamically what its argument types are to construct
an operation \texttt{Object}. After this, our \texttt{ServiceWrapper} takes this serialized request and
builds an introspective call frame. What makes this inefficient is that we must convert twice between \texttt{SelfPortrait's}
representation of method calls and \texttt{Tuscany's}. Changing \texttt{Tuscany's} implementation, the call sequence would
require at most one introspective method call and would, therefore, be much more efficient.

\begin{table}[ht]
\centering
\caption{Division method call}
\label{tab:table2}
\begin{tabular}{ll}
\hline\noalign{\smallskip}
Kind of method call & result\\
\noalign{\smallskip}
\hline
\noalign{\smallskip}
Original & 15360000\\
New with direct method calls & 50000\\
New with proxy method calls & 160770000\\
\hline
\end{tabular}

\end{table}

% TODO discuss code quality improvements

%\begin{itemize}
% \item Compare code quality
% \item Method call performance comparison
% \item Component setup cost
% \item Memory overhead
% \item Autowiring?
%\end{itemize}
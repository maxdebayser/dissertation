In this work we have shown that it is possible to develop native components that are independent of the
underlying infrastructure. We have shown how dependency injection can be used to cut certain kinds
of source-level dependencies resulting in less physical dependencies between modules. Native components
are useful in more constrained environments or when a greater power efficiency is needed. In addition
a middleware as described in this dissertation could ease the integration of native legacy code.

Of course a barrier to a more widespread adoption of native components for service-based systems is precisely
the fact that they have to be compiled for every computer architecture they are to be deployed to. However
there a some interesting developments in that direction. For example, Google's Portable Native Client (\texttt{pNaCL}) \cite{pNaCl}
allows to deploy \texttt{C++} programs under the form of \texttt{LLVM}'s bytecode that is the locally JIT-compiled to native code.
This technology could potentially be reused to be the base of C++ application servers. Another work in this
direction is \texttt{dlSBRT} by Al-Gahmi and Cook \cite{Al-Gahmi}. They describe a mechanism to provide low-level services
to applications such as meta-data, call-graph instrumentation or more conventional services such an LDAP
directory services. These services can intercept events in the application lifetime such as the resolution of
symbols. This could be used for example, by a security manager service to prevent the linking to a symbol
that represents a more privileged functionality, such as writing to a file. \texttt{dlSBRT} is built as an extension
operating system's dynamic linker to provide a more dynamic and configurable environment.
We see these two works and ours as complimentary in building a more flexible environment for native services.
\texttt{pNaCl} provides the necessary portability required to support multiple deployment environments. \texttt{dlSBRT} could
be used to provide more dynamic infrastructure services. For example, \texttt{dlSBRT} could be a great mechanism to load
\texttt{SelfPortrait} meta-data into the running application, by intercepting the references to meta-data symbols, only
the meta-data that is actually used would be loaded, reducing the meta-data memory footprint. Conversely, our
work in dependency injection could be used to inject \texttt{dlSBRT} services into applications in a way that is more
configurable and without source-level dependencies.

Another potentially interesting experiment would be to apply the work of Dubey \emph{et al.} to \texttt{SCA C++} with \texttt{D.I} \cite{Dubey}.
They describe the work that was necessary to create a real-time component framework starting with \texttt{CCM} and a \texttt{ARINC-653},
a real-time operating system. On one hand real-time components usually require fast and predictable response times and therefore
are usually developed in low-level languages. On the other hand, the use of \texttt{CCM} introduces a lot of complexity and source-level
dependencies. Therefore it would be interesting to verify how far we can go with the dependency injection approach considering
that these components have strong requirements on the quality of the infrastructure services.

Finally, although \texttt{XML} is a good language for the declarative configuration of applications, despite being overly verbose, it
is too rigid when a more dynamic behavior is required. As demonstrated by Cerquira \cite{Cerqueira00} a lightweight scripting language with
dynamic typing such as \texttt{Lua} \cite{Lua} can be very useful for gluing together components, even when written for different component
frameworks and in different languages. \texttt{Lua} supports a very clean declarative syntax when needed but also has support for control
flow statements that could allow to write composition files with conditional statements. For example instead of maintaining two
composition files for slightly different environments, as is required with \texttt{SCA's} composition language, with \texttt{Lua} the configuration
script could sense the environment to configure the application accordingly. This is why we have also written a \texttt{Lua} binding
for the \texttt{SelfPortrait API}. Indeed with this binding we can extend Cerqueira's approach to purely local components written
in different languages. With the \texttt{LuaJ} binding and ours, \texttt{Lua} has access to both \texttt{C++} and \texttt{Java} dynamic proxies and could
be used to bridge the communication between \texttt{Java} and \texttt{C++} objects.